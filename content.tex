\subsection{Introduction}


Transcranial electrical and magnetic stimulation  (TES and TMS, respectively) have been increasingly used in studies over the last decades and have been found to
alter and enhance cognitive processes
\supercite{Walsh2000,Lage2016,Paulus2016}. Furthermore, deep brain stimulation (DBS) has shown remarkable results in the treatment of tremor symptoms in 
Parkinsons disease \supercite{Collomb-Clerc2015} and also great potential 
in the treatment of psychiatric disorder such as obsessive-compulsive disorder \supercite{Alonso2015}.
However, despite this growing success in clinical settings, a principled understanding of the effects of stimulation on the dynamical processes in the brain 
is still lacking and, hence, stimulation parameters and target areas are currently not being optimized in a systematic fashion.    

Therefore, Muldoon et al. \supercite{Muldoon2016} develop a framework to explore the effects of targeted transcranial or deep brain stimulation on overall 
brain dynamics. In their framework they use data-driven computational model based on subject-specific structural connectivity and a nonlinear model of regional
brain activity (the so-called Wilson-Cowan model \supercite{Wilson1972}). Furthermore, they demonstrate that structure-based measures from linear network control theory 
can predict the functional effect of targeted stimulation.

In this work, we present an implementation of the modelling framework from Muldoon et al. \supercite{Muldoon2016} writtem im pure Python, where we exchanged
the model of regional brain activity to a faster, phenomenological model, the FitzHugh Nagumo model \supercite{FitzHugh1961}. We report a \textit{partial/full(?)} 
replication of their results.
\subsection{Methods}

Briefly recapitulate 
\begin{itemize}
 \item 
  FitzHugh Nagumo
 \item
  Oscillatory transition parameters
 \item
  Intraclass correlation coefficient (ICC)
 \item
  Linear network control theory
 \item
  Functional and structural effect, fractional activation
\end{itemize}

\subsection{Reproduction of experiments}

\begin{itemize}
 \item 
  Replicate Figure 2 ( b),c) and d) ), 3 states of the FitzHugh Nagumo and comparison to the Wilson-Cowan in b),
  and for c) our box plots for all subjects, and d) bar plots with our data
  
 \item
  Replicate Figure 3 

 \item
  Replicate Figure 4 c) 
  
 \item
  Replicate Figure 5 a)-d)
  
 \item
  Replicate Figure 6 (?)
\end{itemize}

I don't think we need to replicate Figure 7 (Structure-function landscape)

\subsection{Reimplementation}

\begin{itemize}
 \item 
  Details on the new implementation (packages/dependencies, other stuff?, maybe just a paragraph in the methods section) 
\end{itemize}

 




\subsection{Discussion}

\begin{itemize}
 \item 
  Main similarities and differences between our and original results. Replication: full, partial or not at all?
\end{itemize}
